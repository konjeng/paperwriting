\section{Introduction}\label{sec:introduction}

In today's highly competitive e-market, making an online purchase is not an easy task. With the availability of wide range of products and services, choosing a product becomes  more complicated .  To ease up the task, existing e-commerce websites (e.g. Flipkart, Amazon, Snapdeal etc.) provide various ways of exploring and narrowing down the products to help users in making a better choice. Usually these methods include \textbf{category exploration}, \textbf{product search} and \textbf{feature(or facet) selection}. These methods help in narrowing down product search to a small set of results from which the user can choose the right product. But even if the user has found the right product, he has to make sure that quality and features are good enough as per their specification. This is where they rely on the product reviews. 

A product review generally consists of a star rating followed by a few lines of text describing the user experience for the product. Reviews describe the quality of products with respect to certain key terms. These key terms are referred as \textbf{feature} or \textbf{aspect} of the product. For example, in a camera product review, these features could be ”lens”, “zoom” or “picture quality”. Analyzing these features and reviews help to identify the potential customers behavior as well as opinion on products.

While making an on-line purchase, the feature preference of the product varies from user to user. Users generally focus on some subset of features they are interested in, to make their decision. They go through several reviews to find the relevant information and verify the quality about those features. Searching the relevant information out of a pool of unstructured reviews is very tiresome and difficult due to following limitations in existing systems:
\begin{itemize}
	\item
	Existence of large number of unstructured reviews for a single product making it difficult to go through all the reviews.
	\item Filtering reviews based on set features is not developed in any existing state-of-art-systems.
\end{itemize}

To overcome the first limitation, some researchers propose various text summarization techniques [1–3] for summarizing the opinions associated with different features into a rating score measure. These ratings give information about quality of the feature in terms of positivity or negativity. However, the rating systems have its own pitfalls i.e. it fails to answer the question: “what is good or bad about this specific feature”. It just provides information about “how much good or bad is a feature”.
  Most of the times after reading complete review, user unable to find relevant information which consumes lot of time of user. Due to these limitation user needs a better platform which help them in decision making by providing the relevant information while doing fewer interactions. 

Over past decade, a substantial amount of research have already been done in interactive systems for better information retrieval. Many of them used faceted search systems[2] which leverages product facets(or features) information to narrow down relevant information exposed to the user. Other research work is related exploratory navigation systems[1,3,4,5,6] where user can narrow down the results by iteratively selecting facets organised in hierarchical manner. Significance of these techniques are that it can be used to control information exposed to user at certain instance. Our approach uses similar technique for extracting relevant product information from pool of product reviews.      

Usually, a product review describes quality of product with respect to certain frequent key terms. These key terms are generally referred as features of the product. For example, in mobile product reviews feature can be touch screen, processor, RAM etc. As features are frequent over product reviews and are important to the user, it can be used for extracting the useful information out of the reviews. In past, several feature extracting techniques [7,9,11,12] from unstructured text were proposed by many researchers. Organising and presenting these features can also be challenging as it can exist in large numbers and also depends on various parameters. Hence in this paper, we will address these existing problems and propose a feature based review exploration technique where any user can find relevant reviews by choosing relevant features and enhancing user experience by organising these features into effective manner.