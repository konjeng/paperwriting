\section{Related Work}\label{sec:related-work}

Our work is relevant to two broad categories: $faceted$ $navigation$, $opinion$ $search$. Faceted navigation has been a popular research topic in past decade. Due to various advantages of faceted navigation, most of the researcher came up with faceted navigation on various applications. Karlson\cite{Karlson:2006:FFI:1124772.1124878} proposed faceted based interface for mobile interface called $FaThumb$ to browse large amount of information in mobile. $FaThumb$ uses an hybrid model using both keyword search and hierarchical facet metadata navigation to prune out irrelevant data resulting satisfactory results. Faceted navigation for wikipedia searches were proposed by author Hahn and Bizer\cite{hahn2010faceted}. They arranged various attributes templates called infobox templates into a ontology and provided a search interface over RDF triplet knowledge base to extract the relevant information. In these two faceted navigation, faceted hierarchy is manually built. Building manual faceted ontology for reviews is a challenging task as it comes in abundance and unstructured format. In our work, we will show how these hierarchy is created autimatically.

Other relevant research falls under opinion search\cite{liu2012sentiment}. In opinion search, based on user's query, relevant opinionated documents are extracted from document corpus. Relevant documents are usually recognized using certain set of keywords which represents the features of the documents. In Zhang and Yu's work\cite{Zhang:2008:GMU:1390334.1390405}, they used noun and noun pharase extraction technique\cite{Zhang:2007:RCN:1321440.1321540} to extract these features from user's query and find relavent documents involving these feature and their synonyms using similarity search. Opinioned sentence from documents are later extracted using SVM supervised learning. 

In similar area, in $opinion$ $mining$, various feature extraction techniques are discussed. In Hu's work\cite{hu2004mining} using $association$ $rule$ $mining$ technique, frequent features are extracted from the product review. Popescu\cite{popescu2005opine} cameup with Opine system which extracts product features by associating $point-wise$ $mutual$ $information$ $scores$ to frequent noun phrases using unsupervised learning. Also some author proposed sentence-level feature extraction\cite{baccianella2009multi,ding2008holistic}. These approaches depends on text patterns to extract the product features.

Our work leverages conceptNet\cite{liu2004conceptnet} to build feature ontology tree. ConceptNet database can be viewed as semantic network graph database which represent the real world knowledge and common sense relations between various objects. Each node in conceptNet is a concept which represents a real world object and edges represents semantic relationship between various concepts. ConceptNet can be used in various artificial intelligent and text analytics applications to extract the semantic relationship and domain sensitive information. Example of conceptNet relationship is: 

``lens is part of camera''

``camera is capable of taking picture.''

In first example, ``part of'' is the relationship between concepts lens and camera. Similarly for second example, ``capable of'' is the relationship between camera and picture. Our work is adoptation of Mukhargee's work\cite{mukherjee2013sentiment}. In their work, they proposed feature ontology tree construction algorithm to build feature hierarchy. They have used this tree to efficiently feature and sentiments to user. Our goal is different from them i.e. to solve information overload problem while presenting features to user and enhancing understanding of product attributes and their relationships. Our work is different in two aspects. First is we have proposed \textbf{edge type} concept to built a \textbf{directed feature ontology tree}. In \cite{mukherjee2013sentiment} information of $parent$-$child$ relation is missing due to undirected tree. Other difference is that we chose \textbf{BFS exploration} for our tree expansion model and provided inner class priority order to enhance feature ontology tree construction. BFS exploration provide shortest path to relevant concepts which is beneficial for efficient and consistent tree construction. These aspects will be further explained in section 4.2.   
